\documentclass{article}
\usepackage{amsmath}
\usepackage{amsthm}
\usepackage{amssymb}

\title{Problem Set 2}
\author{Michael}
\date{\today}

\begin{document}
\maketitle

\section*{Question 1}
Let $T$ be the collection of all topological spaces. We define a relation $\sim$ on $T$ by $X \sim Y$ if and only if $X$ is homeomorphic to $Y$. Show that $\sim$ is an equivalence relation.

\begin{proof}
To show that $\sim$ is an equivalence relation, we must prove that it is reflexive, symmetric, and transitive.

\subsection*{Reflexivity: ($X \sim X$)}
We must show that any topological space $X$ is homeomorphic to itself. The identity map, $id_X: X \to X$, is a homeomorphism because it is a bijection, continuous, and its inverse (itself) is continuous. Thus, $X \sim X$.

\subsection*{Symmetry: (If $X \sim Y$, then $Y \sim X$)}
We assume there exists a homeomorphism $f: X \to Y$. Its inverse, $f^{-1}: Y \to X$, is also a homeomorphism because it is a bijection, continuous (by definition of $f$), and its inverse, $(f^{-1})^{-1}=f$, is continuous. Thus, if $X \sim Y$, then $Y \sim X$.

\subsection*{Transitivity: (If $X \sim Y$ and $Y \sim Z$, then $X \sim Z$)}
We assume there exist homeomorphisms $f: X \to Y$ and $g: Y \to Z$. The composite map $h = g \circ f: X \to Z$ is a homeomorphism. It is a bijection and continuous because compositions of bijections/continuous functions are bijections/continuous. Its inverse, $h^{-1} = f^{-1} \circ g^{-1}$, is also continuous as it is the composition of two continuous functions. Thus, the relation is transitive.

Since the relation $\sim$ is reflexive, symmetric, and transitive, it is an equivalence relation.
\end{proof}

\section*{Question 2}
\subsection*{(a) Extend the basis of W to V}
To extend the basis of $W$ to $V$, we first take the basis of $W$ and augment it with the basis vectors of $V$. We then iterate through this augmented list, removing any vector that is a linear combination of the preceding ones, until we have a set of $n$ linearly independent vectors. This new basis for $V$ is then transformed into an orthonormal basis, $\{u_1, \dots, u_n\}$, using the Gram-Schmidt process. We ensure the first $k$ vectors, $\{u_1, \dots, u_k\}$, form an orthonormal basis for $W$.

\subsection*{(b) Compute the coset and find the basis for V/W}
A general element $v \in V$ can be written as $v = v_W + v_{W^\perp}$, where $v_W = \sum_{i=1}^{k} c_i u_i$ is in $W$ and $v_{W^\perp} = \sum_{i=k+1}^{n} c_i u_i$ is orthogonal to $W$.
The coset $v+W$ is computed as:
$$ v + W = (v_W + v_{W^\perp}) + W = v_{W^\perp} + (v_W + W) = v_{W^\perp} + W $$
The basis for the quotient space $V/W$ is the set of cosets of the basis vectors that are orthogonal to $W$:
$$ \text{Basis}(V/W) = \{ u_{k+1} + W, u_{k+2} + W, \dots, u_n + W \} $$

\subsection*{(c) Determine the formula for the dimension of V/W}
The dimension of $V/W$ is the number of vectors in its basis, which is $n-k$. Since $n=\dim(V)$ and $k=\dim(W)$, we have the formula:
$$ \dim(V/W) = \dim(V) - \dim(W) $$

\section*{Question 3}
\subsection*{(a) Find the matrix presentation $M_k$ of $\partial_k$}
The dimensions of the chain groups are $\dim(C_0)=7$, $\dim(C_1)=12$, and $\dim(C_2)=3$. The two non-trivial boundary matrices are $M_2$ ($\partial_2: C_2 \to C_1$) and $M_1$ ($\partial_1: C_1 \to C_0$).

\textbf{The Matrix $M_2$ for $\partial_2: C_2 \to C_1$:}
This is a $12 \times 3$ matrix.
$$ M_2 = \begin{pmatrix} 1&0&0\\ 1&0&0\\ 1&0&0\\ 0&1&0\\ 0&1&0\\ 0&1&0\\ 0&0&1\\ 0&0&1\\ 0&0&1\\ 0&0&0\\ 0&0&0\\ 0&0&0 \end{pmatrix} $$

\textbf{The Matrix $M_1$ for $\partial_1: C_1 \to C_0$:}
This is a $7 \times 12$ matrix.
$$
M_1 = \left[\begin{array}{cccccccccccc}
    1 & 0 & 1 & 0 & 0 & 0 & 0 & 0 & 0 & 1 & 0 & 0 \\
    1 & 1 & 0 & 1 & 0 & 1 & 1 & 0 & 1 & 0 & 0 & 0 \\
    0 & 1 & 1 & 0 & 0 & 0 & 0 & 0 & 0 & 0 & 1 & 0 \\
    0 & 0 & 0 & 1 & 1 & 0 & 0 & 0 & 0 & 0 & 1 & 0 \\
    0 & 0 & 0 & 0 & 1 & 1 & 0 & 0 & 0 & 0 & 0 & 1 \\
    0 & 0 & 0 & 0 & 0 & 0 & 1 & 1 & 0 & 1 & 0 & 0 \\
    0 & 0 & 0 & 0 & 0 & 0 & 0 & 1 & 1 & 0 & 0 & 1
\end{array}\right]
$$
The matrices $M_3$ and $M_0$ are zero maps.

\subsection*{(b) Put $M_k$ in reduced row echelon form}
\textbf{RREF of $M_2$:} Has **3 pivot columns** and **0 non-pivot columns**.
\textbf{RREF of $M_1$:} Has **6 pivot columns** and **6 non-pivot columns**.

\subsection*{(c) Compute the Betti numbers}
Using the formula $\beta_k = \dim(\text{Ker}(\partial_k)) - \dim(\text{Im}(\partial_{k+1}))$:
\begin{itemize}
    \item $\boldsymbol{\beta_2} = \dim(\text{Ker}(\partial_2)) - \dim(\text{Im}(\partial_3)) = 0 - 0 = \mathbf{0}$.
    \item $\boldsymbol{\beta_1} = \dim(\text{Ker}(\partial_1)) - \dim(\text{Im}(\partial_2)) = 6 - 3 = \mathbf{3}$.
    \item $\boldsymbol{\beta_0} = \dim(\text{Ker}(\partial_0)) - \dim(\text{Im}(\partial_1)) = 7 - 6 = \mathbf{1}$.
\end{itemize}

\subsection*{(d) Compute the Euler Characteristic $\chi(K)$}
We compute $\chi(K)$ in two ways:
\begin{enumerate}
    \item \textbf{By counting simplices:}
    $$ \chi(K) = 7 - 12 + 3 = \mathbf{-2} $$
    \item \textbf{Via Betti numbers (Euler-Poincaré):}
    $$ \chi(K) = \beta_0 - \beta_1 + \beta_2 = 1 - 3 + 0 = \mathbf{-2} $$
\end{enumerate}
The results match, confirming our calculations.

\end{document}